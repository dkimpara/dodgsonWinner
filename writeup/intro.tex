\section{Introduction}\label{sec:intro}

    Suppose you are a theoretical computer scientist, wandering in the wilds
of theory-land and you come upon a complexity class.
It has not been seen before.
It seems intuitive and natural.
So you decide to submit it to the complexity zoo \citep{zoo}.
But the worry is whether this class will turn out to capture the complexity
of important real-world problems.
In other words, is anyone or any problems going to come around to your new
section of the zoo and join your party?
After all, there are 545 classes and counting
in the zoo!

This was the case for the class \tp in the mid-1990s.
\tp  or \cc{P}$^{\np}_{\parallel}$, the class of
languages decidable by a polynomial time Turing machine with parallel access to
\np, was introduced in 1983 by \citep{PZ83}.
The location of this class in the polynomial hierarchy is:


 %   $$\sig{p}{1} \cup \pic{p}{1} \subseteq \tp \subseteq \Delta^{p}_2
 %   \subseteq \sig{p}{2} \cap \pic{p}{2},~~~\text{or}$$

  %  $$\np \cup \cc{coNP} \subseteq \cc{P}^{\np}_{\parallel}
   % \subseteq \cc{P}^{\np} \subseteq \np^{\np} \cap \cc{coNP}^{\np}$$ \\


By the mid 1990s, the theoretical importance of \tp was recognized in complexity theory.
Klaus W. Wagner established half a dozen characterizations of \tp  \citep{wag90},
several complete problems and a toolkit for establishing \tp-hardness.
\citet{hem87} and \citet{ksw87} showed that \tp was equivalent to the class of
problems that can be solved by $\mathcal{O}(\log(n))$ sequential Turing queries to \np.
Furthermore, if \np contains some \tp-hard problem, then the polynomial hierarchy
collapses to \np.
However, these connections and complete problems
lived in the pure theory section of the zoo and did not yet have the appeal of
problems from ``real world" settings.

Along came the Dodgson winner problem, invented in 1876 by Charles Lutwidge
Dodgson, better known under the pen name of Lewis Carroll.
\citet{exactdodgson} proved that this problem was \tp-complete.
This was the first ``real-world" problem proven complete for the class \tp.

Dodgson's election system takes the following form.
An election is a finite number of voters, who each cast a linear order
over a common finite set of candidates.
Note that linear orders are ``tie-free".
The winner is determined by whichever candidate is closest to being a Cordorcet
winner, a criteria used by other election systems.
A Condorcet winner is a candidate $a$ who for every other candidate $b$, is
preferred to $b$ by strictly more than half of the voters.
We naturally want election systems to be Condorcet-consistent, i.e. the system
has the property where if $a$ is a Condorcet winner, $a$ is the one and only
winner in the election.
Dodgson's election system is Condorcet-consistent \citep{handbookcss}

The winner(s) in a Dodgson election is defined as the candidate(s) who are
the ``closest" to being Condorcet winners.
The winners are the candidates that have the lowest Dodgson score.
The Dodgson score of a candidate $a$ is the smallest number of
sequential exchanges of adjacent candidates in preference orders such that
after those exchanges $a$ is a Cordorcet winner.

Note that it is remarkable that we find \tp-complete problems that were defined
100 years before complexity theory itself existed.
Dodgson winner is also extremely natural when compared with other complete
problems in this class such as determining if the maximum size clique in a
graph is of odd size (odd-max-clique).
In this project we will use \algprobm{Comp-SAT}, %todo format
which was recently shown to be complete for this class \citep{compsat}


The rest of this project presents the theory of and a practical algorithm for
the Dodgson winner problem.
We present slightly modified proof of completeness of
the Dodgson winner problem based on new results for the class \tp.
Then we present, implement, and examine a heuristic algorithm
that is self-knowingly correct for most practical instances of the problem.



