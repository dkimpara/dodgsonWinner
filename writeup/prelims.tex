\section{Preliminaries}

\subsection{Problem Setting}

\begin{defn}{Dodgson Triple}\\
    A triple, $\langle C, c, V \rangle$, where $C$ is the set of candidates
    $1, \dots , m$, a candidate $c \in C$, and a set of $n$ strict
    (ie. irreflexive and anti-symmetric) preference orders,
    one per voter, over all candidates in $C$.
\end{defn}



\begin{defn}{Condorcet Winner}\\
    In an election, with a set of candidates $C$ and $n$ votes or strict preference
    orders $V$, a candidate $a \in C$ is a Condorcet Winner if for every other
    candidate $b$, $a \succ b$ by strictly more than half of the voters.

\end{defn}

\begin{defn}{Dodgson Score}\label{def:dscore}\\
    First define a switch as an exchange of two adjacent preferences in the
    preference order of one voter.
    Then, the Dodgson Score of a candidate is the smallest number of sequential
    switches needed to make the candidate a
    Condorcet winner.
    The Dodgson Score of any Cordorcet Winner is 0.
    We denote the Dodgson score of a Dodgson triple as $Score(\dtriple)$
\end{defn}

\begin{problem}{\algprobm{DodgsonScore}} \\
    \tab \textbf{Instance:} $k \in \mathbb{N}$.
        A Dodgson Election and Candidate $\langle C,c,V \rangle$. \\
    \tab \textbf{Decide:} Is the Dodgson Score of candidate $c$ less than or
        equal to $k$?
\end{problem}

\begin{problem}{\algprobm{DodgsonWinner}} \\
    \tab Instance: A Dodgson Election and Candidate $\langle C,c,V \rangle$ \\
    \tab Decide: Is $c$ a winner of the election?
    In other words, does $c$ have the minimum Dodgson Score in the election?

\end{problem}

\begin{problem}{\algprobm{Comp-M}} \\
    \tab Instance: A pair $\langle A,B \rangle$ of sets of \np-hard decision problems
    in $M$.\\
    \tab Decide: Is the number of yes-instances in $A$
    greater than or equal to the number of yes-instances in $B$.
\end{problem}

\begin{problem}{\csat} \citep{compsat}\\
    \tab Instance: A pair $\langle A,B \rangle$ of sets of 3CNF formulas.\\
    \tab Decide: Is the number of satisfiable formulas in $A$
    greater than or equal to the number of satisfiable formulas in $B$.
\end{problem}

\subsection{Complexity Classes and Definitions}

\begin{defn}{\tp:} \\
    The class of problems solvable with polynomial-time parallel access to an
    \np oracle.
    This is equivalent to $\mathcal{O}(\log(n))$ sequential queries to an \np
    oracle \citep{hem87, ksw87}.
\end{defn}

\begin{theorem}\label{thm:csat} \citep{compsat}
    \csat is \tp-complete.
\end{theorem}



