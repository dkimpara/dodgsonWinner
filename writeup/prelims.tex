\section{Preliminaries}

\subsection{Problem Setting}

We formally introduce the Dodgson Winner problem, or the decision problem
of checking whether a particular candidate is a Dodgson winner in a
Dodgson election.
First, we formally define a Dodgson Election, or an instance of the decision
problem \dwin, which we
will define later.

\begin{defn}{Dodgson Triple.}\\
    A triple, $\langle C, c, V \rangle$, where $C$ is the set of candidates
    $1, \dots , m$, a candidate $c \in C$, and a set of $n$ strict
    (ie. irreflexive and anti-symmetric) preference orders,
    one per voter, over all candidates in $C$.
\end{defn}

\begin{defn}{Preference relation.} \\
    For vote $v$ and candidates $c,d \in C$. $c<_{v}d$ means that in vote
$v$, $d$ is preferred to $c$ and $c\prec_v d$ means that $c<_v d$ and
$\nexists e ~ \in C ~ s.t. ~ c <_v e <_v d$.
\end{defn}

A natural criterion for the winner of an election
is the concept of a Condorcet Winner.
This concept is used in other election systems such as Young's election, which
is related to Dodgson's system.

\begin{defn}{Condorcet Winner} \citep{condorcet1785essay}.\\
    In an election, with a set of candidates $C$ and $n$ votes or strict preference
    orders $V$, a candidate $a \in C$ is a Condorcet Winner if for every other
    candidate $b$, $a > b$ by strictly more than half of the voters.
\end{defn}

However, many elections will not have a Condorcet winner, so Dodgson's idea
was that whichever candidate was the ``closest'' to being a Cordorcet winner
should be the winner.
It is also possible to have multiple winners, all with equal scores (closeness)
in this system.
A candidates closeness to being a Cordorcet winner is defined as the Dodgson Score
of that candidate.

\begin{defn}{Dodgson Score.}\label{def:dscore}\\
    First define a switch as an exchange of two adjacent preferences in the
    preference order of one voter.
    Then, the Dodgson Score of a candidate is the smallest number of sequential
    switches needed to make the candidate a
    Condorcet winner.
    The Dodgson Score of any Cordorcet Winner is 0.
    We denote the Dodgson score of a Dodgson triple as $Score(\dtriple)$
\end{defn}

This notion of ``closeness'' is based on finding the minimum edit distance
between the initial votes and a region in the space of all votes.
The edits are sequential and defined as adjacent exchanges in the preference orders.
The candidates with the minimum Dodgson Scores are the winners of the
Dodgson election.

We now define the formal decision problems with which we will examine from a
theory of computation lens.
The first problem, \dscore, will be used to show that the main problem,
\dwin, is \tp-complete.

\begin{problem}{\algprobm{DodgsonScore}.} \\
    \tab \textbf{Instance:} $k \in \mathbb{N}$.
        A Dodgson Election and Candidate $\langle C,c,V \rangle$. \\
    \tab \textbf{Decide:} Is the Dodgson Score of candidate $c$ less than or
        equal to $k$?
\end{problem}

\begin{problem}{\algprobm{DodgsonWinner}.} \\
    \tab Instance: A Dodgson Election and Candidate $\langle C,c,V \rangle$ \\
    \tab Decide: Is $c$ a winner of the election?
    In other words, does $c$ have the minimum Dodgson Score in the election?

\end{problem}


\subsection{\tp~in the Polynomial Hierarchy}

Here we formally introduce the class \tp, for which
we will show \dwin~is complete for.
This was the first natural ``real-world'' problem that was shown to be complete
for this class.
As briefly talked about in the introduction, \tp~and the
general classes $\Theta_{k}^P$ hold a particular position
in the polynomial hierarchy.
$\Theta_{k}^P$ however is not usually talked about in regular definitions of
the polynomial hierarchy so we will also see in this section how they fit in.

First we introduce the polynomial hierarchy (PH).
PH generalizes the classes \np~ and \cc{coNP}.
PH is formed by the classes $\Sigma_k^P$, $\Pi_k^P$, and $\Delta_k^P$.
It is defined as follows: $\Sigma_0^P = \Pi_0^P = \Delta_0^P = \cc{P}$ and
for all $k \geq 1$,
$\Sigma_k^P = \cc{NP}^{\Sigma_{k-1}^P}$,
$\Delta_k^P = \cc{P}^{\Sigma_{k-1}^P}$, and
$\Pi_k^P = \cc{co}\Sigma_{k-1}^P$.
Note that
$\Sigma_1^P = \cc{NP}^{\Sigma_{0}^P} = \cc{NP}^{P} = \cc{NP}$
and $\Pi_1^P = \cc{co}{\Sigma_{0}^P} = \cc{coNP}^{P}$.
The union of all the classes in the PH is denoted \cc{PH}.
An open problem in computer science is whether or not
the polynomial hierarchy collapses.
The polynomial hierarchy is said to collapse if for some $k$, $\Sigma_k^P$ or $\Pi_k^P = \cc{PH}$.
It is strongly believed that the
polynomial hierarchy does not collapse.

\citet{PZ83} proposed that $\Theta_{k}^P = \cc{P}^{\Sigma_{k-1}^P
[O(\log n)]}$ to be included in the standard definition of PH as well,
where the bound in the brackets represents the bound on the allowed
number of oracle calls to the oracle class preceding the brackets.
Note that \citet{hem87} showed that this is equivalent to parallel
access to $\Sigma_{k-1}^P$.
\citet{compsat} poses that $\Theta_0^P = \cc{P}$ and observe that
$\Theta_1^P = P$.
Also $\Theta_{k}^P$ is a deterministic class, since the
machine making oracle calls is deterministic.
Thus $\Theta_{k}^P = \cc{co}\Theta_{k}^P$.
We state the generalized relationship of the classes in terms of
$k$:

$$\sig{P}{k} \cup \pic{P}{k} \subseteq
    \Theta_{k+1}^P \subseteq \Delta^{P}_{k+1}
    \subseteq \sig{P}{k+1} \cap \pic{P}{k+1}$$

Now our specific problem, \dwin, is complete for \tp:

\begin{defn}{\tp:} \\
    The class of problems solvable with polynomial-time parallel access to an
    \np~oracle.
    This is equivalent to $\mathcal{O}(\log(n))$ sequential queries to an
\np~oracle \citep{hem87, ksw87}.
\end{defn}

While the classes $\Sigma^P_k$ and $\Pi^P_k$ in the polynomial hierarchy
have their complete problem as quantified boolean formulas with $k-1$
alternations of quantifiers (\algprobm{QBF$_k$} or \algprobm{QSAT$_k$}),
the class \tp~did not yet have such a characterization by some formulation
of \algprobm{SAT} until recently.
The complete problem for the class \tp, with this similar flavor, is
\csat, introduced in \citet{csatintro}.
In the same work, a general characterization for the classes
$\Theta_{k+1}^P$ was also proved, given by the following problem:


\begin{problem}{\algprobm{Comp-PH$_k$}} \label{thm:cph}  \\
    \tab \textbf{Instance}: Let $A_1$ and $A_2$ be either both $\Sigma_k^P$-complete
or $\Pi_k^P$-complete problems.
    Then let $\langle A,B \rangle$ be sets of instances of $A_1$ and
    $A_2$, respectively.\\
    \tab \textbf{Decide}: Is the number of yes-instances in $A$
    greater than or equal to the number of yes-instances in $B$.
\end{problem}

\begin{theorem}\label{thm:validk} \citep{compsat}
    \algprobm{Comp-PH$_k$} is $\Theta_{k+1}^P$-complete.
\end{theorem}

For a formulation via \algprobm{QBF$_k$}, let $A_1$ and $A_2$ be
both the
same appropriate \algprobm{QBF$_k$} problem that is complete for
$\Sigma_k^P$ or $\Pi_k^P$.
Note that \citet{compsat} state this latter problem as complete,
calling it \algprobm{Comp-Valid$_k$}, however this result is based on Theorem
\ref{thm:validk}.
The completeness of this problem makes sense because
$\Theta_{k+1}^P$ is intuitively the
class of problems solvable with parallel oracle access to the $k$th level of
the polynomial hierarchy.
For the class \tp, of which \dwin~is complete, the canonical problem is \csat.

\begin{problem}{\csat or \algprobm{Comp-Valid$_1$}} \citep{compsat}\\
    \tab \textbf{Instance}: A pair $\langle A,B \rangle$
of sets of 3CNF formulas.\\
    \tab \textbf{Decide}: Is the number of satisfiable formulas in $A$
    greater than or equal to the number of satisfiable formulas in $B$.
\end{problem}


\begin{theorem}\label{thm:csat} \citep{compsat}
    \csat is \tp-complete.
\end{theorem}



